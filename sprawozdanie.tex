\documentclass[a4paper]{article}
% Kodowanie latain 2
%\usepackage[latin2]{inputenc}
\usepackage[T1]{fontenc}
% Można też użyć UTF-8
\usepackage[utf8]{inputenc}

% Język
\usepackage[polish]{babel}
% \usepackage[english]{babel}

% Rózne przydatne paczki:
% - znaczki matematyczne
\usepackage{amsmath, amsfonts}
% - wcięcie na początku pierwszego akapitu
\usepackage{indentfirst}
% - komenda \url 
\usepackage{hyperref}
% - dołączanie obrazków
\usepackage{graphics}
% - szersza strona
\usepackage[nofoot,hdivide={2cm,*,2cm},vdivide={2cm,*,2cm}]{geometry}
\frenchspacing
% - brak numerów stron
\pagestyle{empty}

% dane autora
\author{Łukasz Kulczycki}
\title{Sprawdzian PWI}
\date{\today}

\begin{document}
\maketitle

\section{Zadanie}
\begin{verbatim}
ssh-keygen
ssh-copy-id -i ~/mykey mpyzik@pwi.ii.uni.wroc.pl
ssh 'mpyzik@pwi.ii.uni.wroc.pl'
\end{verbatim}

\section{Zadanie}
\begin{verbatim}
ls
ls -R
mkdir lukasz_kulczycki
touch lukasz_kulczycki.txt
nano lukasz_kulczycki.txt
less lukasz_kulczycki.txt
seq 0 7 100 >> lukasz_kulczycki.txt
exit
scp mpyzik@pwi.ii.uni.wroc.pl:/home/mpyzik/lukasz_kulczycki/testy/lukasz_kulczycki.txt 
/mnt/c/Users/^_^/Desktop/sprawdzian/
ssh 'mpyzik@pwi.ii.uni.wroc.pl'
\end{verbatim}

\section{Zadanie}
\begin{verbatim}
mkdir dane
cd dane
hexdump /dev/urandom | head > plik1.txt
hexdump /dev/urandom | head > plik2.txt
hexdump /dev/urandom | head > plik3.txt
hexdump /dev/urandom | head > plik4.txt
hexdump /dev/urandom | head > plik5.txt
touch concat.txt
cat plik1.txt >> concat.txt
cat plik2.txt >> concat.txt
cat plik3.txt >> concat.txt
cat plik4.txt >> concat.txt
cat plik5.txt >> concat.txt
grep "^0.*\([a-f0-9][a-f0-9]\)\1$" concat.txt >> output.txt
wc -l output.txt concat.txt
exit
\end{verbatim}

\end{document}